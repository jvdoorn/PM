\documentclass[10pt]{article}
\parindent=0pt
\usepackage{fullpage}
\frenchspacing
\usepackage{microtype}
\usepackage[english,dutch]{babel}
\usepackage{listings}
% Er zijn talloze parameters ...
\lstset{language=C++, showstringspaces=false, basicstyle=\small,
  numbers=left, numberstyle=\tiny, numberfirstline=false, breaklines=true,
  stepnumber=1, tabsize=4,
  commentstyle=\ttfamily, identifierstyle=\ttfamily,
  stringstyle=\itshape}

\title{Product}
\author{Julian van Doorn}

\begin{document}

\selectlanguage{english}

\maketitle

\section{Explanation}
This program is based on the Python version I wrote (together with Tobias Eikelenboom) for PRna. It aims to provide the functionality described in the assignment 'Product'.

\section{Time}
Translating the Python code to c++, getting used to c++ and debugging took around 3 hours. Tidying everything up, testing edge cases and writing the report another hour. The time spent on the original program was around 8 hours, a lot of which was spent during work colleges.

\section{Analysis}
I think the program could be written more efficiently using basic math operations. However, it would of been more effort to find that method than to use my existing code. Had I had more time I would of optimised the program. Moving from Python to c++ was sometimes frustrating as c++ doesn't have as many built in 'ready-to-go' functions as Python does. Par example to simply convert a string or character to lowercase.

\section*{Code}
\lstinputlisting{main.cpp}

\end{document}
